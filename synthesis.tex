@book{
  gulwani2017program,
  author    = {Gulwani, Sumit and Polozov, Alex and Singh, Rishabh},
  title     = {Program Synthesis},
  booktitle = {Foundations and Trends in Programming Languages},
  year      = {2017},
  month     = {August},
  abstract  = {Program synthesis is the task of automatically finding a program in the underlying programming language that satisfies the user intent expressed in the form of some specification. Since the inception of AI in the 1950s, this problem has been considered the holy grail of Computer Science. Despite inherent challenges in the problem such as ambiguity of user intent and a typically enormous search space of programs, the field of program synthesis has developed many different techniques that enable program synthesis in different real-life application domains. It is now used successfully in software engineering, biological discovery, computer-aided education, end-user programming, and data cleaning. In the last decade, several applications of synthesis in the field of programming by examples have been deployed in mass-market industrial products. This survey is a general overview of the state-of-the-art approaches to program synthesis, its applications, and subfields. We discuss the general principles common to all modern synthesis approaches such as syntactic bias, oracle-guided inductive search, and optimization techniques. We then present a literature review covering the four most common state-of-the-art techniques in program synthesis: enumerative search, constraint solving, stochastic search, and deduction-based programming by examples. We conclude with a brief list of future horizons for the field.},
  publisher = {NOW},
  url       = {https://www.microsoft.com/en-us/research/publication/program-synthesis/},
  pages     = {1-119},
  volume    = {4}
}