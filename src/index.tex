\newif\ifcommentenabled \commentenabledtrue
\newcommand{\TS}[1]{\ifcommentenabled\textcolor{red}{TS: #1}\fi}
\newcommand{\TK}[1]{\ifcommentenabled\textcolor{blue}{TK: #1}\fi}

\documentclass[12pt, a4paper, titlepage]{report}
\usepackage[cache=false]{minted} % highlight programming languages
\usepackage[dvipdfmx]{graphicx}
\usepackage[nottoc,numbib]{tocbibind}
\usepackage[utf8]{inputenc}
\usepackage{amsmath}
\usepackage{color}
\usepackage{enumerate}
\usepackage{hyperref}
\usepackage{mathptmx}
\usepackage{pdfpages} % PDF inclusion
\usepackage{verbatim} % comment environment
\hypersetup{
  colorlinks = true,
  linkcolor = cyan
}

% Title Page
\title{Bachelor Thesis \\ TODO: TITLE}
\author{
  03190413 Takemaru Kadoi
  \\[1cm]
  {\small Supervisor: Prof. Masahiro Fujita},
  {\small Advisor: Assistant Prof. Taro Sekiyama}
  \\[1cm]
  {\small University of Tokyo, Department of Information and Communication Engineering}
}
\date{\today}

\begin{document}

% front page
\includepdf[pages={-}]{images/sotsuronFront.pdf}

% header
\maketitle
\newpage
\tableofcontents
\newpage

\chapter*{Acknowledgment}
This research was supervised by Prof. Fujita at the University of Tokyo and supported by Assistant Prof. Sekiyama at National Institute of Technology. \\
I thank my colleagues at the University of Tokyo, Miyasaka-san, Koike-san, Yu-san, Nagasawa-san, and Yi-san.
% TODO: add twitter accounts
I also show my gratitude to my good friends, hnkz, Akio, Saeki, Arahata, Hori, Naito, Chen, and Sakura.

% overview of the research topic
\chapter{Abstract} \label{section:abstract}
In recent years, several tools have been developed to enhance the productivity of programmers.
In particular, tools for generating code have attracted a lot of attention, and program synthesis is a research field that supports this development.
Program synthesis requires specifications with which users tell a computer what they want to achieve.
In the case of logical formulae \TS{expressions $\rightarrow$ formulas / formulae} \TK{use "formulae" instead of "expressions".}, it is
difficult for programmers who are not skilled in math while its strictness contributes to accurate code generation. \TS{The advantages of logical formulas should be described.} \TK{add an explanation about its strictness.}
With natural language, computers may not be able to understand the meaning, and the intention may not be conveyed correctly even though programmers can find it easy to write specifications. \TS{What are the advantages of specifications in natural languages?} \TK{add an explanation about its human friendly character.}
This paper examines the use of a type and effect system as a specification that is easy to understand by both humans and computers alike. The types are properties of data and can be used to describe the input and output of a function. The effects, which can explain the internal processes of a function, represent a transition in a state, such as the value of a variable, shared memory between threads, and I/O operations.
\TS{What are types and effects?} \TK{add explanations}
Types are familiar to programmers who use statically typed languages.
Many users are unfamiliar with effects, but the effects \TS{What "they" means is a bit ambiguous. It is clearer to say "effects".} \TK{use "effects".}can describe the internal behavior of programs \TS{Here is the first place you mention functions, so it sounds abrupt.  It may be natural to say "code" or "programs"} \TK{use "code".}and are easy to understand if designed properly.
By utilizing type and effect, I can efficiently explore candidate code and demonstrate that the intended code will be generated. I will demonstrate some specifications and an example of a program that has been generated based on these specifications.
\TS{(DONE)How do you synthesize programs with types and effects?  What experiments are (planed to be) conducted?  How about results?} \TK{describe how to synthesize and what to do.}

In Chapter \ref{section:introduction}, I discuss the motivation for this research, what program synthesis is, and existing methods  The details of the existing methods can be discussed in the related work section, which should be  put before the conclusion.
I give examples of existing researches in Chapter \ref{section:relatedWork}.
I explain my proposed method in Chapter \ref{section:method}.
I present in Chapter \ref{section:experiment} the results obtained by the proposed method and evaluate them.
Chapter \ref{section:conclusion} concludes with a summary of the results and evaluation, as well as future issues.

\chapter{Introduction}\label{section:introduction}
\TS{Introduction should include a summary of what this thesis achieves.  The details of the existing methods can be discussed in the related work section, which should be  put before the conclusion.} \TK{move "Related Work"(previous "Existing methods") to after this chapter and add section "Achievement" in this chapter}
  \section{Motivation}
    The purpose of this research is to enable people to program efficiently and safely in the current software-centric world that demands for many lines of secure code.
    I tackle this problem by using a type-effect system and a technique of program synthesis.
    Bodik tackled insists that this issue is tractable from non-programmer and programmer perspectives. \cite{bodik:2015}
    I, however, focus only on a programmer point of view.

    Today, more and more people are developing software.
    It is believed that an increasing number of people are writing software proportionally.
    As a result of the current situation, the demand for secure software is also increasing.
    The type systems of recent programming languages, such as Rust, Scala, and Elm, are strict.
    The type theory supports the type systems.
    Software development is facilitated by programming languages that are based on rigorous theories.
    Additionally, the increasing demand illustrates the need for more time-efficient and productive programming.
    The program synthesis was developed in response to this need.
    I am motivated by these trends in my research.
    My goal is to help people develop reliable and fast software through type theory and program synthesis.
    Additionally, I would like to incorporate the effect system into my research.
    It is an informative and descriptive system for programming languages.
    This system allows us to describe the internals of a function.

  \section{Program Synthesis}
    In this section, I would like to explain what program synthesis is for those new to this field. Gulwani, Polozov, and Singh summerizes in their survey paper \emph{"Program Synthesis"} \cite{gulwani:2017} summarizes it. 
  \section{Achievement} % contribution of this paper and intro to technical part

\chapter{Related Work}\label{section:relatedWork}
% TODO: get texts from progress report(https://www.overleaf.com/project/613c638a684e51edccdfc7b2)
  \section{Enumerative Search}
  \section{Constraint Solving}
  \section{Stochastic Search}
  \section{Deduction-based Programming}

% my approach to the problem
\chapter{Method}\label{section:method}
  \section{Syntax}
    \begin{minted}[breaklines, linenos]{ocaml}
      let _ = print_endline "hello world"
    \end{minted}
  \section{Semantics}
    \subsection{Static Semantics} % type-checking rules
    \subsection{Dynamic Semantics} % evaluation rules
  \section{Type and Effect System}
    \cite{pierce:2002}
    \subsection{Type System}
    \subsection{Effect System}
  \section{Synthetic Process}

% my approach to the problem
\chapter{Experiment}\label{section:experiment}
\section{Result}
\section{Consideration}

% qualitative result and remaining problems/future work
\chapter{Conclusion}\label{section:conclusion}

\bibliographystyle{unsrt}
\bibliography{
  bib/msSynthesis,
  bib/synthesisOp,
  bib/tapl,
}

\end{document}
